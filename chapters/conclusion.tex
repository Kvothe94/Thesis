\chapter{Conclusion and Future Work}\label{c:conclusion}
In this work we have tried to enhance the reliability of the myocontrol system used currently at DLR using Formal Verification techniques: in particular we have tried to tackle the problem of activation overshooting. Our studies brought us to the development of two different automated procedures which can enhance the reliability of our system of interest: the SMT-Repair process, which uses an SMT solver in order to give formal guarantees on the results, and the PDO-Repair process, which uses statistical methods that don't give formal guarantees on the results. After the development of the above mentioned automated procedures we have designed two experiment in order to compare the performance of the standard controller and those of the two controllers repaired using our automated procedures. In the second of the two experiment we dampened the prediction $f_{A}$ in order to elicit the activation overshooting and therefore to be able to verify the efficacy of our repair procedures. In the first experiment we didn't dampen the prediction in order to be sure that our repair processes at least don't worsen the performance of the controller in the standard applications.
We have analysed the results from the two experiments in section \ref{sec:exp-results}: as can be seen from the analysis of the Reaching Rates of both experiments (see: \ref{subsub:first-RR}, \ref{subsub:second-RR}) the automated procedures effectively manage to enhance the reliability of the myocontrol system to the problem of overshooting. Moreover from the analysis of the Success Rates (see: \ref{subsub:first-SR}, \ref{subsub:second-SR}) it is clear that not only the performance of the controller doesn't worsen in the standard applications after the repair processes but also that while the standard controller is almost useless when the participant is encouraged to use more force, the repaired ones still preserve a certain level of usability. The analysis done on TCTs and TITs (see: \ref{subsub:first-TCT}, \ref{subsub:first-TIT}, \ref{subsub:second-TCT}, \ref{subsub:second-TIT}) appears to hint to an higher ease of use for the SMT-Repaired controllers but we didn't manage to obtain decisive results about this hypothesis. In general the SMT-Repair process appears to present an higher temporal overhead, mainly due the use of the SMT-solver, compared to the PDO-Repair process (see: \ref{subsub:first-TTR}, \ref{subsub:second-TTR}). All in all it appears that both repair processes manage to substantially increase the reliability of the myocontrol system and therefore we have managed to successfully tackle the problem of interest.