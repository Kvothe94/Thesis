\begin{abstract}
Myocontrol is a hot subtopic of assistive robotics, in particular it is one of the so-far unsolved hurdles in upper-limb prosthetics. It is about swiftly, naturally and reliably converting biosignals, non-invasively gathered from an upper-limb amputated subject, into control commands for an appropriate self-powered prosthetic device.
Despite decades of research, traditional surface electromyography cannot yet detect the subject's intent to an acceptable degree of reliability, that is, enforce an action exactly when the subject wants it to be enforced.\\\\
In this work we tackle one of the subproblems related to myocontrol reliability, namely activation overshooting, and show that Formal Verification can indeed be used to mitigate it at an acceptable computational cost. Eighteen intact subjects were engaged in two Target Achievement Control tests in which a standard myocontrol system was compared with two "repaired" ones, one using a simple non-formal technique, enforcing no guarantee of safety, and the other using Satisfiability Modulo Theories (SMT) technology to rigorously enforce it. The experimental results indicate that both repaired systems exhibit an improved reliability by reducing activation overshooting. Using the SMT-based system only requires a modest increase in the required computational resources.
\end{abstract}