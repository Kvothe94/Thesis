\section{Motivations}\label{sec:motivations}
As testified in \cite{biddiss2007upper} the mean rejection rate for electric prostheses is 35\% for the pediatric population and 23\% for the adult population and one of the critical factor for the rejection is the unsatisfactory state of the available technology [\cite{biddiss2007upperfact}].
In particular one of the lacking aspect pointed out in \cite{biddiss2007upperfact} and, more recently, in \cite{castellini2016upper} was the ease of control of the prostheses: whereas the machine learned controllers have grown more and more refined [\cite{Strazzulla2017}] the reliability of this kind of controllers is still the bottleneck for the passage from the research community to the industrial one.
The problem of the insufficient reliability of machine learning system has become increasingly interesting for the research community in the last few years: although machine learning systems are becoming more and more common in industrial application, they present limited application in safety/security critical domain due to their limited reliability. In order to enhance the reliability of machine learning systems the research community has tried different approaches: in particular we are interested in the approach which uses formal methods, and in particular formal verification, in order to analyse and eventually repair machine learning models. In \cite{leofante2018automated} a thorough presentation of the state of the art of the above mentioned approach can be found.
Even if the research done on this approach consider mostly neural networks as machine learning models of interest and, as far as we know, there hasn't been any temptative to apply this kind of methods in the domain of prosthesis control, we believe that the application of formal methods could truly enhance the reliability of our machine learned myocontroller and, consequently, improve the confidence on deploying the controller and lower the rejection rate.