\section{Context}\label{s:context}
As testified in \cite{ZIEGLERGRAHAM2008422} "One in 190 Americans is currently living with the loss of a limb. Unchecked, this number may double by the year 2050": this kind of statistic can easily express how much prosthetic technologies are important nowadays and how big is the market for them.
By virtue of the above-mentioned high demand of prosthesis, technological research in the prosthetic domain in the last decades has been very active: ideally amputees would need, as far as possible, prosthesis as functional as real limbs and a great deal of research as been done to try to enhance the performance, the comfort and the appearance of prosthetic limbs. Sadly, even with all the effort done from the scientific community, we are still far from developing this kind of prosthetic limbs: in particular, even if relatively dexterous prostheses are commercially available, reliable prosthetic control is still an open problem. As testified in \cite{castellini2016upper}, detecting the patient's intent and transforming it into effective control signals is still a largely open problem. During the last few years machine learning has become more and more common as control method: various machine learning model has been used to extrapolate control policy from data provided from disparate type of sensors. Even so the control system is still the bottleneck for the diffusion of machine learned controlled prosthesis in the daily life of the standard amputee:in
One of the most commonly used sensor is the electromyographic sensor, which allows to measure the electrical activity of muscles: this kind of sensor owes its popularity to its (relatively) low cost and to the fact that it can be used without the need of invasive surgical procedures. Although the control methods for prosthesis using EMG signals are constantly improving, from \cite{Zecca2002} to \cite{Strazzulla2017}, . 