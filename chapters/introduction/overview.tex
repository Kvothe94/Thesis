\section{Overview}\label{sec:overview}
This thesis is structured as follows:

Chapter \ref{ch:background} briefly presents the background topics, such as an introduction to myoelectric control, formal methods, in particular formal verification, Satisfiability and Satisfiability Modulo Theories. Moreover we present the machine learning models of interest, in particular Ridge Regression (RR) and Ridge Regression with Random Fourier Features (RR-RFF).

Chapter \ref{ch:preliminaries} introduces the myocontrol system currently used at DLR, in particular the hardware used with the system and the most important characteristic of the interaction between the hardware and the software components. We also present the SMT solver we have decided to use in this work and its most important features.

Chapter \ref{ch:problem-definition} presents our formal definition of the problem of interest and how we have proceeded in our study in order to develop a feasible solution. We also show the structure of our algorithms and an example of execution of the same. We then explain the main differences between the different algorithms and their features.

Chapter \ref{ch:exp-result} presents the experiments we have designed in order to analyse the efficacy of our algorithms, it explains the choices we have made in the designing of the experiments and it shows the experimental protocol we have followed. Furthermore it presents the experimental results we have found from the statistical analysis of the data collected during the two experiments. 

Chapter \ref{ch:conclusion-future}  points out achieved results and future work directions.