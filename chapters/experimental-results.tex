\chapter{Experimental Results}\label{ch:exp-result}
\section{Introduction}\label{sec:exp-intro}
In this chapter we will present the experiments we have done in order to study the effects of our repair procedures on the performances of the learning model. We will explain the choices we have done in the design of the experiments, the experimental protocol and then we will show the results we have obtained.
The measure of performances in the prosthetic domain is almost always done in experiment with human participants: given my inexperience in human subject research, all the choices for the experiment design and the results analysis are based on \cite{de2017human}.
\section{Experiment Design}\label{sec:exp-design}
To compare the performances of the original learning model and of the repaired models we have chosen to do two different experiments: the only difference between the two experiments is given by the value of a single parameter, therefore we have chosen to present the design of the two experiment at the same time.
Both our experiments are instances of the Target Achievement Control (TAC) test for prosthetic hands [\cite{simon2011target}], in which the participant is asked to have a virtual upper limb match a feasible target configuration of the same limb which is visually presented to the participant, from now on we will call a single instances of this kind of test \textit{task}.
In our cases each participants will need to perform a series of tasks using the standard myocontroller, one repaired using the SMT-Repair procedure and one repaired using the PDO-Repair procedure.
Given the time we had to carry out the experiments and the need to keep contained the number of participant required for the results analysis we decided to follow the \textbf{within-subject design} for our experiment, e.g. each participant need to complete a set of task \textit{on each controller}. The results yielded from an experiment using the within-subject design are susceptible to order and carryover effect, therefore is necessary to implement counterbalancing. Counterbalancing aims to keep under control order and carryover effect by letting the participants undergo the various conditions in different orders: in our case the participant will complete the task on the different learning machines in various order. For example the first participant will complete the set of task in a certain order first on the unrepaired controller and then on the other machines, whereas another participant will undergo the same set of task in a different order first on the SMT-Repaired controller and then on the other machines.
In order to avoid that knowledge about the experiment could modify the performances of the participants in the experiment we decided to use \textbf{masking}, e.g. the participants didn't know that they were performing on different controllers and, as consequence, they didn't know which controller were using at any time.
\subsection{Experimental Setup}\label{subsec:exp-setup}
The experimental setup was common to both the experiments: it consisted of a Myo bracelet (see section \ref{sec:interactivemyocontrol}) and two 3D hand models displayed on a computer screen. This two hand model realistically mimic the motions of a human wrist and hand: one is used to provide the visual reference, e.g. it is controlled by the software, whereas the other is used to show the predicted motions as provided by the bracelet and by the chosen controller. We call the first hand model \textit{Stimulus} and the second one \textit{Prosthesis}.
\subsection{Participants}\label{subsec:exp-participants}
As participants for the first experiment we found six intact humans, of which one was female and the others were males, all between 21 and 32 years old. In the other experiment we had twelve intact humans as participants, four were females and six were males, all between 23 and 27 years old. Before beginning the experiment, we clearly explained to each participant, both orally and in writing, that no health risk was involved in the experiment.  Each participant signed an informed consent form. The experiment was previously approved by the internal committee for data protection of the Institution where the experiments took place, and it followed the World Medical Association’s Declaration of Helsinki.
\subsection{Experiment Parameters}\label{subsec:exp-parameters}
For both experiments we chose a tolerance threshold of 15\% for the goal to be reached, 1.5 seconds for the required target dwelling time and 15 seconds for the timeout time. In the second experiment we dampened the prediction $f_A$ in order to hopefully elicit overshooting: when the participant would try to reach the maximum activation for $t_A = 1$ they would only see a prediction equals to $f_A = 0.75$ and therefore she would be encouraged to increase her force to $t_A > 1$.  
\section{Experimental Protocol}\label{sec:exp-protocol}
To each participant was assigned a series of 36 TAC tasks (i.e., hand/wrist configuration to mimic on the screen) on the three different controllers (Unrepaired, SMT-Repaired, PDO-Repaired), therefore each participant needed to complete 12 TAC tasks using each one controller. The tasks covered the three active action of interest: power grasp, wrist flexion and wrist extension; clearly the number of task for each action and for each controller is the same (e.g. 4 tasks for each action and for each machine).
In order to achieve counterbalancing (see section \ref{sec:exp-design}) the order of both the tasks and the controllers was randomised for each participant, whereas the number and type of task was the same for each participant.
In either experiments the participant sat comfortably in front of the computer screen with the Myo bracelet wrapped around her forearm. She was instructed to keep her forearm with an angle of 45 degree and her elbow resting on the armrest.
We then explained to the participant that she would be required to complete a training session for the prosthesis before facing a series of 36 task, during which she would be required to guide the Prosthesis in following the reference shown by the Stimulus as close as possible using the signal from the movements of her own arm.
At the beginning of the training session the Stimulus was shown on the screen and we explained to the participant that the Stimulus would perform a series of movement and that she should mimic what the Stimulus was doing with her own arm. At the end of the explanation the data gathering phase began: the Stimulus played once each action of interest (rest, RE; power grasp, PW; wrist extension, EX; wrist flexion, FL) and while the participant followed the movements of the Stimulus the observation were collected. After the end of the previous phase the Prosthesis would appear on the screen and the series of 36 task would began.
\section{Experimental Results}\label{sec:exp-results}
\subsection{Performance Measures and Statistics}
For each task we decided to report the following measures of performance:
\begin{itemize}
    \item \textbf{Success Rate} (\textbf{SR}): this measure is the fraction of task successfully completed, an instance of this measure is computed for each participant as the number of tasks completed successfully by the participant divided for the total number of tasks faced by that participant.
    \item \textbf{Time to Complete Task} (\textbf{TCT}): this measure is the time needed to complete a task. This measure is computed separately for each task.
    \item \textbf{Time In Target} (\textbf{TIT}): this measure is the total time the participant managed to stay in the target before the timeout or before the completion of the task. This measure is computed separately for each task.
    \item \textbf{Reaching Rate} (\textbf{RR}): this measure is the fraction of time during which the output $f_A$ was higher than the minimum acceptable value, e.g. the percentage of time the participant actually managed to reach the desired activation value.
    We chose to consider this measure in order to verify whether a failure was really due to overshooting or due to another problem like interference.
\end{itemize}