\chapter{Experimental Results}\label{ch:exp-result}
\section{Introduction}\label{sec:exp-intro}
In this chapter we will present the experiments we have done in order to study the effects of our repair procedures on the performances of the learning model. We will explain the choices we have done in the design of the experiments, the experimental protocol and then we will show the results we have obtained.
The measure of performances in the prosthetic domain is almost always done in experiment with human participants: given my inexperience in human subject research, all the choices for the experiment design and the results analysis are based on \cite{WinterDodou17}.
\section{Experiments Design}\label{sec:exp-design}
To compare the performances of the original learning model and of the repaired models we have chosen to do two different experiments: the only difference between the two experiments is given by the value of a single parameter, therefore we have chosen to present the design of the two experiment at the same time.
Both our experiments are instances of the Target Achievement Control (TAC) test for prosthetic hands [\cite{simon2011target}], in which the participant is asked to have a virtual upper limb match a feasible target configuration of the same limb which is visually presented to the participant, from now on we will call a single instances of this kind of test \textit{task}.
In our cases each participants will need to perform a series of tasks using the standard myocontroller, one repaired using the SMT-Repair procedure and one repaired using the PDO-Repair procedure.
Given the time we had to carry out the experiments and the need to keep contained the number of participant required for the results analysis we decided to follow the \textbf{within-subject design} for our experiment, e.g. each participant need to complete a set of task \textit{on each controller}. The results yielded from an experiment using the withing-subject design are susceptible to order and carryover effect, therefore is necessary to implement counterbalancing. Counterbalancing aims to keep under control order and carryover effect by letting the participants undergo the various conditions in different orders: in our case the participant will complete the task on the different learning machines in various order. For example the first participant will complete the set of task in a certain order first on the unrepaired controller and then on the other machines, whereas another participant will undergo the same set of task in a different order first on the SMT-Repaired controller and then on the other machines.
In order to avoid that knowledge about the experiment could modify the performances of the participants in the experiment we decided to use \textbf{masking}, e.g. the participants didn't know that they were performing on different controllers and, as consequence, they didn't know which controller were using at any time.